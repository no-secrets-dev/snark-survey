\begin{table}[!t]
\caption{Comparison of work done by pairing-based (S)NARKs. 
$n$ represents the number of circuit gates; $\mathbb{G}_1$ and 
$\mathbb{G}_2$ represent group elements; $\mathbb{F}$ represents field elements; $\mathbf{P}$ represents pairing operations. 
In prover/verifier work columns, $\mathbb{G}_i$ and $\mathbb{F}$ refer to elliptic curve group scalar multliplications in $\mathbb{G}_i$ 
and field element multiplications in $\mathbb{F}$, respectively. An asterisk implies the method is not fully succinct. Where more fine-grained 
source group information is easily discerned, $\mathbb{G}$ implies the elements / operations could be in either source group.
Where more fine-grained information is not available or easily comparable in a standardized manner, we resort to asymptotic terms.}
\label{tbl:snark}
\begin{tabular}{|l|l|l|p{3.2cm}|l|c|c|l|}
\hline
\toprule
method & CRS size (asymp.) & $\mathcal{P}$ work (asymp.) & proof size & $\mathcal{V}$ work & universal & updatable & assumptions \\ \hline\toprule
\midrule
GGPR13 & $O(n) \mathbb{G}$ & $O(n) \mathbb{G}$ & $9 \mathbb{G}$ & $14 \mathbf{P}$ & No & No & q-PKE, q-PDH \\ \hline
PGHR13 & $O(n) \mathbb{G}$ & $O(n) \mathbb{G}$ & $8 \mathbb{G}$ & $11 \mathbf{P}$ & No & No & $q$-PKE, $q$-PDH \\ \hline
Groth16 & $O(n) \mathbb{G}_1, O(n) \mathbb{G}_2$ & $O(n) \mathbb{G}_1$ & $2 \mathbb{G}_1, 1 \mathbb{G}_2$ & $3 \mathbf{P}$ & No & No & $q$-type, GGM \\ \hline
GKM+18 & $O(n^2) \mathbb{G}$ & $O(n) \mathbb{G}_1$ & $2 \mathbb{G}_1, 1 \mathbb{G}_2$ & $5 \mathbf{P}$ & Yes & Yes & $q$-type, KOE \\ \hline
MBKM19 & $36n \mathbb{G}_1$ & $273n \mathbb{G}_1$ & $20 \mathbb{G}_1, 16 \mathbb{F}$ & $13 \mathbf{P}$ & Yes & Yes & AGM \\ \hline
Gab19* & $2n \mathbb{G}_1$ & $8n \mathbb{G}_1$ & $6 \mathbb{G}_1, 4 \mathbb{F}$ & $5 \mathbf{P}$ & Yes & Yes & AGM \\ \hline
GWC19 & $3n \mathbb{G}_1, 2 \mathbb{G}_2$ & $11n \mathbb{G}_1$ & $7 \mathbb{G}_1, 6 \mathbb{F}$ & $2 \mathbf{P}, 16 \mathbb{G}_1$ & Yes & Yes & AGM \\ \hline
GW21 & $9n \mathbb{G}_1, 2 \mathbb{G}_2$ & $35n \mathbb{G}_1$ & $4 \mathbb{G}_1, 15 \mathbb{F}$ & $2 \mathbf{P}, 5 \mathbb{G}_1$ & Yes & Yes & AGM \\ \hline
CHM+19 & $(4n +2) \mathbb{G}_1$ & $22n \mathbb{G}_1$ & $13 \mathbb{G}_1, 8 \mathbb{F}$ & $2 \mathbf{P}$ & Yes & Yes & AGM \\ \hline\bottomrule
\bottomrule
\bottomrule
\end{tabular}
\end{table}
