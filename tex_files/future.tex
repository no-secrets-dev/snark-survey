\section{Future Research}

\subsection{Post-Quantum SNARKs}
\noindent Though multiple of the SNARKs discussed are practical from a performance standpoint, quantum computers may render them useless from a security standpoint in the near future. As a result, there is a strong effort to develop efficient enough post-quantum SNARKs. To this end, many works \cite{starks, ligero, fractal, spartan} use hashing-based cryptography relying on hash function collision-resistance, an assumption believed to hold against quantum adversaries. The common denominator is to use hashing-based polynomial proximity testing methods such as FRI \cite{FRI}, which involve merkle tree commitments to Reed-Solomon codewords of witness polynomials. While such approaches are believed to be post-quantum secure, they pay a hefty price by using a cryptographihc primitive affording no algebraic structure to its outputs. The most interesting direction addressing these issues involves SNARK constructions using lattice-based cryptography. Albrecht et al. \cite{lattice1} made some progress in this direction via a lattice-based SNARK with logarithmic-time verification. As it is operating over a completely different algebraic object, this method uses a completely different set of security assumptions than those discussed prior -- namely, generalizations of the short integer solution (SIS) problem. Future work that quantifies the security assumptions in this area and decreases prover/verifier complexity would bring SNARKs with efficiency imposed by algebraic structure and post-quantum security to reality, which the blockchain ecosystem would eventually accept given sufficient tooling and support.

\subsection{Proof aggregation \& Distributed Proving}
\noindent As one would expect, proof generation remains the performance bottleneck in most SNARKs (pairing-based or not). In a world where computational resources are quite asymmetrically distributed (DeFi being no exception), it would be sensible to be able to outsource more intensive computations within the same proof to another party; naturally they would include a proof that such computation was done correctly. \textbf{Is there any work on this right now}. A similar idea emerges when we consider distributed proving of a statement by segmenting the \textit{statement} instead of the \textit{proof components}. Each delegated prover would then have to prove that a given subcircuit has an assignment which produces some desired value, then all of these proofs would get aggregated along with a proof that the aggregation was performed correctly.\\

\subsection{Standardized benchmarks for pairing-based methods (and others)}
\noindent Current SNARK benchmarks lack standardization by the computation being verified.\\

\subsection{Integration of optimizations to pairing-based cryptography}
\noindent optimize miller function\\

\subsection{Formal verification of pairing-based systems}
\noindent maybe\\

\subsection{Post-quantum pairing-based SNARKs (how?)}
\noindent Could be done using pairings for isogeny-based cryptography?\\

\subsection{Extending pairings to higher-degree operations}
\noindent nothing\\
